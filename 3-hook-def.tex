\chapter{Определение хука}
\label{hook-0}

\begin{flushleft}
LSM (Linux Security Modules) предоставляет множество хуков, которые позволяют модулям безопасности внедрять свои политики и проверки в различные части ядра Linux. Эти хуки охватывают широкий спектр операций, таких как управление процессами, доступ к файловой системе, сетевые операции и другие.
	
Существует множество точек входа для внесения своего хука (\href{https://elixir.bootlin.com/linux/v6.1/source/security/selinux/hooks.c#L7116}{\underline{все selinux хуки}}). Но для нашей задачи выбран хук cred\_prepare:

Хук cred\_prepare вызывается при подготовке структуры учетных данных (credentials) для нового процесса. Это важный этап, потому что учетные данные включают в себя информацию о правах доступа и идентификацию процесса (например, идентификаторы пользователя и группы).

Благодаря этому хуку, мы можем отловить процесс исполнения интересующего системного вызова. Он может быть не только sys\_fchmod, но и, практически, любым другим (не для всех системных вызовов нужна структура учетных данных).
\end{flushleft}