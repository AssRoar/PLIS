\section{Теория. LSM.}
\label{theory-lsm-0}

\begin{flushleft}
	
	\paragraph{Введение}
	
	LSM (Linux Security Modules) — это инфраструктура в ядре Linux, предназначенная для поддержки различных моделей безопасности. Она позволяет разработчикам создавать и интегрировать собственные модули безопасности, предоставляя механизмы для контроля доступа и безопасности на уровне ядра. Вот подробный обзор технологии LSM.
	
	\paragraph{Цели LSM}
	
	\begin{itemize}
		\item[•] Предоставление универсального API для реализации различных политик безопасности.
		\item[•] Обеспечение механизмов для контроля доступа к системным ресурсам, таким как файлы, процессы, сети и память.
		\item[•] Позволяет внедрять модули безопасности без необходимости модификации ядра.
	\end{itemize}
	
	\paragraph{Архитектура и компоненты LSM}
	
	eBPF (extended BPF) — это значительно расширенная версия оригинального BPF:
	\begin{itemize}
		\item[1] LSM Hooks (крючки LSM):
			\begin{itemize}
				\item[-] Специальные точки в коде ядра, где могут быть вызваны функции безопасности.
				\item[-] Позволяют модулям безопасности внедрять свои проверки и решения в стандартные операции ядра, такие как создание файлов, доступ к памяти, межпроцессное взаимодействие и сетевые операции.
			\end{itemize}
			
		\item[2] LSM Модули:
		\begin{itemize}
			\item[-] Наборы правил и политик безопасности, которые используют LSM hooks для контроля доступа.
			\item[-] Примеры популярных LSM модулей:
				\begin{itemize}
					\item[•] SELinux (Security-Enhanced Linux): предоставляет гибкую и мощную модель контроля доступа на основе меток.
					\item[•] AppArmor: пспользует профили для ограничения возможностей процессов на уровне пути к файлу.
					\item[•] И другие.
				\end{itemize}
		\end{itemize}
	\end{itemize}
	
	LSM — это мощный и гибкий механизм, предоставляющий разработчикам и администраторам возможность внедрения и управления различными моделями безопасности в Linux.
	
\end{flushleft}