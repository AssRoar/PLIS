\chapter{Введение}
\label{intro:0}
\begin{flushleft}

Для написания и тестирования программы были установлены:

\begin{itemize}
	\item[-] Заголовочные и исходные файлы ядра Linux версии 6.1;
	\item[-] libbpf пакет версии 1.4 (с оптимизацией llvm);
\end{itemize}
Настройки окружения:
\begin{itemize}
	\item[-] Операционная система Arch Linux;
	\item[-] Ядро версии 6.9.3;
	\item[-] gcc версии 14.1;
\end{itemize}


В дальнейшем понадобиться дампить объектные файлы для чтения их инструкций, для этого применяется:
\begin{verbatim}
	llvm-objdump-14 -d *.bpf.o
	или
	llvm-objdump -d *.bpf.o
\end{verbatim}

\ \\
Модуль (kernel security module) deny\_unshare применяется для контроля системного вызова unshare, при создании нового пространства.\\

\ \\
Для использования модуля необходимо:
\begin{itemize}
	\item[1.] Собрать все необходимые пакеты для компиляции BPF программ.
	\item[2.] Собрать исполняемый файл.
	\item[3.] запустить исполняемый файл:
	\begin{verbatim}
					./<file_name>
	\end{verbatim}
\end{itemize}


Для прекращения работы программы нужно послать сигнал SIGINT/INT (например, просто в терминале процесса нажать Ctrl+C).
\\ \

Для тестирования работоспособности программы применяются попытки создания нового пространства.
\end{flushleft}

