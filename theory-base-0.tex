\chapter{Теория. Основы.}
\label{theory-base-0}
\begin{flushleft}
	
Данный документ не затрагивает развитие мысли вредоносности неконтролируемого создания пространств любым пользователем, поэтому ограничимся тем, что это приводит к большой проблеме безопасности для НОВЫХ пользователей подобных систем. Для примера применения приведу возможность "масштабирования" подобного пространства для того, что бы новые пользователи вносили свои данные в уже захваченную среду/систему. Так же, благодаря этому можно "выйти" из ограниченного окружения, что подробно описано в  \href{https://book.hacktricks.xyz/linux-hardening/privilege-escalation/docker-security/namespaces/user-namespace}{\textcolor{blue}{этой}} статье. \\

Начну с пояснения используемых инструментов:
\begin{itemize}
	\item[1.] Технология BPF (см. главу \underline{\nameref{theory-bpf-0}})
	\item[2.] Технология LSM (см. главу \underline{\nameref{theory-lsm-0}})
\end{itemize}
Для дальнейшего чтения стоит ознакомиться с соответствующими вышему главами.

Для работы этих двух инструментов была использована внутренняя возможность LSM и BPF, которые изначально позволяют удобно внедрять BPF инструкции к LSM хукам.

\end{flushleft}