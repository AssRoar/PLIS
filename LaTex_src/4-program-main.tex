\section{Работа программы. Main.}
\label{program-main}

\begin{flushleft}

\begin{verbatim}
    #include <bpf/libbpf.h>
    #include <unistd.h>
    #include "security.skel.h"
    
    int main(int argc, char *argv[])
    {
        struct security_bpf *skel;
        int err;
        skel = security_bpf__open_and_load();
        if (!skel) { 
            fprintf(stderr, "Ошибка генерации BPF каркаса.\n");
            goto cleanup;
        }
        err = security_bpf__attach(skel);
        if (err) {
            fprintf(stderr, "Ошибка внедрения BPF инструкций.\n");
            goto cleanup;
        }
        printf("BPF программа успешно загружена.\n");
        for (;;) {
            fprintf(stderr, ".");
            sleep(1);
        }
        cleanup:
        security_bpf__destroy(skel);
        return err;
    }
\end{verbatim}
    Структура security.c файла достаточно проста. Она использует возможности libbpf (в частности, утилиту bpftool) для генерации всего необходимого, в том числе и байт-инструкций. После чего этим же инструментом вносит программу в ядро и входит в бесконечный цикл.
\end{flushleft}