\section{Теория. BPF.}
\label{theory-bpf-0}

\begin{flushleft}
	
\paragraph{Введение}

	Технология BPF (Berkeley Packet Filter) изначально была разработана для фильтрации пакетов в сетевых приложениях, таких как tcpdump. Однако со временем BPF значительно эволюционировал, особенно в контексте ядра Linux, превратившись в мощный механизм, используемый для различных целей, включая мониторинг производительности, сетевую безопасность и диагностику.
	
\paragraph{Исходная концепция}

\begin{itemize}
	\item[•] BPF был представлен в 1992 году как эффективный способ фильтрации пакетов в операционной системе BSD.
	\item[•] Исходный BPF включал в себя байт-код, который выполняется в виртуальной машине внутри ядра, что позволяло быстро и эффективно обрабатывать сетевые пакеты.
	\item[•] Изолированность среды выполнения BPF инструкций (байт-код, описанный выше).
\end{itemize}

\paragraph{eBPF: Расширенный BPF}

eBPF (extended BPF) — это значительно расширенная версия оригинального BPF:
\begin{itemize}
	\item[•] Введение в ядре Linux: eBPF был интегрирован в ядро Linux начиная с версии 3.15.
	\item[•] Расширение функциональности: eBPF позволяет выполнять произвольный код в ядре, что открывает возможности для различных типов мониторинга и обработки данных.
	\item[•] Поддержка новых типов карт eBPF (BPF maps), которые являются ключевым элементом для хранения и обмена данными между программами eBPF и пользовательским пространством.
	\item[•] Расширенная поддержка проб eBPF. Пробы (probes) eBPF используются для динамической вставки точек отслеживания в код ядра.
	\item[•] BPF Type Format (BTF) — это новый формат, позволяющий программам eBPF получать доступ к информации о типах данных в ядре Linux.
	\item[*] Остальные инновации нас интересовать будут меньше.
\end{itemize}


\newpage

\paragraph{Архитектура и компоненты eBPF}

\begin{itemize}
	\item[1.] Программы eBPF:
		\begin{itemize}
			\item[•] Написаны на высокоуровневом языке, таком как C, и затем компилируются в байт-код eBPF.
			\item[•] Загружаются в ядро через системные вызовы и могут быть прикреплены к различным точкам в ядре (например, к сетевым событиям, системным вызовам, трассировочным точкам).
		\end{itemize}
		
	\item [2.] BPF-карты (BPF maps):
		\begin{itemize}
			\item[•] Представляют собой структуры данных в ядре, используемые для обмена информацией между программами eBPF и пользовательским пространством.
			\item[•] Поддерживают различные типы данных, такие как хэш-таблицы, массивы, счетчики.
		\end{itemize}
		
	\item[3.] Системные вызовы для работы с eBPF:
		\begin{itemize}
			\item[•] bpf(): основной системный вызов для загрузки, управления и взаимодействия с программами и картами eBPF.
		\end{itemize}
\end{itemize}


Технология BPF, а особенно её расширенная версия eBPF, представляет собой мощный инструмент для мониторинга, диагностики и управления системами на уровне ядра. Благодаря своей гибкости и производительности, eBPF находит широкое применение в современных вычислительных системах, предоставляя разработчикам и администраторам новые возможности для управления и оптимизации работы систем.

\end{flushleft}