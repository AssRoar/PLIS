\chapter{Работа программы}
\label{program}


\begin{flushleft}
	Теперь мы перейдем к описанию работы программы. Процесс может быть логически разделен на следующие ключевые компоненты:
	
	\begin{itemize}
		\item[1.] \textbf{Написание хука: } Это начальный этап, включающий разработку функции хука, которая будет интегрироваться в ядро для выполнения определенных задач безопасности.
		\item[2.] \textbf{Генерация низкоуровневых инструкций: } На этом этапе происходит трансляция кода хука в низкоуровневые инструкции, совместимые с внутренней архитектурой ядра.
		\item[3.] \textbf{Модификация сдвигов памяти для обращений к данным ядра: } Этот шаг включает корректировку смещений памяти при обращениях к данным ядра. В данном контексте следует ознакомиться с технологией Co-Re (Compile Once - Run Everywhere), реализованной в библиотеке libbpf, которая позволяет динамически адаптировать программы eBPF к разным версиям ядра.
		\item[4.] \textbf{Замена/регистрация нового хука: } После генерации и корректировки кода хука, производится его замена или регистрация в системе, что позволяет новому хуку получать данные ядра изнутри.
		\item[5.] \textbf{Безопасное отключение хука: } Обеспечение безопасного отключения хука, что включает его корректное удаление из системы и освобождение всех связанных ресурсов. Этот шаг важен для поддержания стабильности и безопасности системы.
	\end{itemize}
	
	Пункт 1 остается за нами. Пункты с 2 по 4 выполняются при помощи встроенных возможностей библиотеки libbpf, которая позволяет уменьшить объем работы в десятки раз. Пункт 5 является не столько задачей сделать что-то, сколько задачей не делать ничего: у bpf программ есть особенность внесения и изъятия - для их однократной отработки достаточно внести их в ядро с определенным триггером, но для постоянной нужны триггер (в нашем случае - это вызов хука) и дескриптор (это не совсем так, но для простоты объяснения оставлю) файла, в котором хранятся инструкции. В нашем случае, файлом являются данные работающей программы, что при её завершении автоматически удалятся, что приведет к изъятию bpf инструкций из ядра.
\end{flushleft}