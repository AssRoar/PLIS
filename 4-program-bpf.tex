\section{Работа программы. BPF.}
\label{program-bpf}

\begin{flushleft}
    В случае с кодом хука всё интереснее: 
\begin{verbatim}
    #include <linux/bpf.h>
    #include <linux/capability.h>
    #include <linux/errno.h>
    #include <linux/sched.h>
    #include <linux/types.h>
    #include<linux/kernel.h>
    #include <bpf/bpf_tracing.h>
    #include <bpf/bpf_helpers.h>
    #include <bpf/bpf_core_read.h>
    
    #define X86_64_UNSHARE_SYSCALL 272
    #define UNSHARE_SYSCALL X86_64_UNSHARE_SYSCALL
    
    typedef unsigned int gfp_t;
    
    struct pt_regs {
        long unsigned int di;
        long unsigned int orig_ax;
    } __attribute__((preserve_access_index));
    
    typedef struct kernflagsel_cap_struct {
        __u32 cap[_LINUX_CAPABILITY_U32S_3];
    } __attribute__((preserve_access_index)) kernel_cap_t;
    
    struct cred {
        kernel_cap_t cap_permitted;
    } __attribute__((preserve_access_index));
    
    struct task_struct {
        unsigned int flags;
        const struct cred *cred;
    } __attribute__((preserve_access_index));
    
    char LICENSE[] SEC("license") = "GPL";
    SEC("lsm/cred_prepare")
    int BPF_PROG(handle_cred_prepare, struct cred *new, const struct cred *old, gfp_t gfp, int ret)
    {
        if (ret) return ret;
        
        struct pt_regs *regs;
        struct task_struct *task;
        int syscall;
        unsigned long flags;
        
        task = bpf_get_current_task_btf();
        regs = (struct pt_regs *) bpf_task_pt_regs(task);    
        syscall = regs -> orig_ax;
        
        if (syscall != UNSHARE_SYSCALL) return 0;
        
        flags = PT_REGS_PARM1_CORE(regs);
        if (!(flags & CLONE_NEWUSER)) {
            return 0;
        }
        
        return -EPERM;
    }
\end{verbatim}

Вдаваться в подробности каждой структуры не стану - это очень глубокая лужа, которая со стороны не кажется опасной. интереснее здесь \_\_attribute\_\_((preserve\_access\_index)) - макрос для компилятора и для дальнейшей релокации данных, который как раз магически определяет неоднозначное определение полей. Это относится к Co-Re технологии. В самом начале хука мы возвращаем ret - возвращенное значение предыдущего такого же хука. Это необходимо из-за того, что при внесении нашего хука, он добавляется в очередь на вызов хуков. То есть он может быть не единственным, из-за чего, для сохранения результата предыдущих хуков, все должны передавать результат дальше по списку (и да, на этом месте также можно реализовать вредоносную программу, которая попросту будет нарушать это, что приведет к неработоспособности других хуков безопасности). regs - регистры. Они хранят все внешние параметры для системного вызова. В целом, остальное всё понятно.

\end{flushleft}